% Options for packages loaded elsewhere
\PassOptionsToPackage{unicode}{hyperref}
\PassOptionsToPackage{hyphens}{url}
%
\documentclass[
]{article}
\usepackage{amsmath,amssymb}
\usepackage{iftex}
\ifPDFTeX
  \usepackage[T1]{fontenc}
  \usepackage[utf8]{inputenc}
  \usepackage{textcomp} % provide euro and other symbols
\else % if luatex or xetex
  \usepackage{unicode-math} % this also loads fontspec
  \defaultfontfeatures{Scale=MatchLowercase}
  \defaultfontfeatures[\rmfamily]{Ligatures=TeX,Scale=1}
\fi
\usepackage{lmodern}
\ifPDFTeX\else
  % xetex/luatex font selection
\fi
% Use upquote if available, for straight quotes in verbatim environments
\IfFileExists{upquote.sty}{\usepackage{upquote}}{}
\IfFileExists{microtype.sty}{% use microtype if available
  \usepackage[]{microtype}
  \UseMicrotypeSet[protrusion]{basicmath} % disable protrusion for tt fonts
}{}
\makeatletter
\@ifundefined{KOMAClassName}{% if non-KOMA class
  \IfFileExists{parskip.sty}{%
    \usepackage{parskip}
  }{% else
    \setlength{\parindent}{0pt}
    \setlength{\parskip}{6pt plus 2pt minus 1pt}}
}{% if KOMA class
  \KOMAoptions{parskip=half}}
\makeatother
\usepackage{xcolor}
\usepackage[margin=1in]{geometry}
\usepackage{graphicx}
\makeatletter
\def\maxwidth{\ifdim\Gin@nat@width>\linewidth\linewidth\else\Gin@nat@width\fi}
\def\maxheight{\ifdim\Gin@nat@height>\textheight\textheight\else\Gin@nat@height\fi}
\makeatother
% Scale images if necessary, so that they will not overflow the page
% margins by default, and it is still possible to overwrite the defaults
% using explicit options in \includegraphics[width, height, ...]{}
\setkeys{Gin}{width=\maxwidth,height=\maxheight,keepaspectratio}
% Set default figure placement to htbp
\makeatletter
\def\fps@figure{htbp}
\makeatother
\setlength{\emergencystretch}{3em} % prevent overfull lines
\providecommand{\tightlist}{%
  \setlength{\itemsep}{0pt}\setlength{\parskip}{0pt}}
\setcounter{secnumdepth}{-\maxdimen} % remove section numbering
\ifLuaTeX
  \usepackage{selnolig}  % disable illegal ligatures
\fi
\IfFileExists{bookmark.sty}{\usepackage{bookmark}}{\usepackage{hyperref}}
\IfFileExists{xurl.sty}{\usepackage{xurl}}{} % add URL line breaks if available
\urlstyle{same}
\hypersetup{
  pdftitle={writeup},
  pdfauthor={Chaeyeong},
  hidelinks,
  pdfcreator={LaTeX via pandoc}}

\title{writeup}
\author{Chaeyeong}
\date{2024-11-30}

\begin{document}
\maketitle

\hypertarget{research-design}{%
\section{Research Design}\label{research-design}}

Research Question How is financial inclusion (e.g., account ownership,
saving, borrowing, and digital payments) related to poverty levels
across countries with complete data for 2014, 2017, and 2021? Research
Objectives To examine the correlation between financial inclusion
variables and poverty levels. To control for economic factors like GDP
and assess their moderating effect on poverty. To evaluate temporal
changes in the relationship between financial inclusion and poverty over
the three years. Key Variables Dependent Variable: Poverty Level:
poverty\_headcount\_ratio Independent Variables (Financial Inclusion):
per\_account\_ownership\_poor: Percentage of poor individuals with a
financial account. per\_saving\_poor: Percentage of poor individuals
saving in financial institutions. per\_borrowing\_poor: Percentage of
poor individuals borrowing from financial institutions.
per\_digital\_payment\_poor: Percentage of poor individuals making
digital payments. Control Variables: Economic Growth: gdp
(log-transformed for better interpretation). Empirical Strategy Model
Specification: Use a fixed-effects (FE) panel regression model to
control for unobserved, time-invariant country-specific characteristics
(e.g., geography, culture). Where:

\[
\text{Poverty\_Headcount\_Ratio}_{it} = \beta_0 + \beta_1(\text{Account\_Ownership}_{it}) + \beta_2(\text{Savings}_{it}) + \beta_3(\text{Borrowing}_{it}) + \beta_4(\text{Digital\_Payments}_{it}) + \beta_5(\log(GDP_{it})) + \alpha_i + \gamma_t + \epsilon_{it}
\] i: Country. t: Year. 𝛼𝑖: Country fixed effects. 𝛾𝑡: Year fixed
effects. 𝜖𝑖𝑡: Error term

\begin{enumerate}
\def\labelenumi{\arabic{enumi}.}
\tightlist
\item
  Balanced Panel Data Analysis Focus your analysis on the 47 countries
  with data for all three years. This design enables you to apply
  fixed-effects or random-effects models for a balanced panel, which
  provides robust insights into the temporal relationships between
  financial inclusion variables and poverty across years. Advantages:
  Robust temporal and cross-sectional comparisons. Controls for
  country-level unobserved heterogeneity. Disadvantages: Reduces the
  sample size, potentially limiting the generalizability of results.
  Implementation:
\end{enumerate}

필터링하는 경우 (3개 연도 데이터로 제한): 장점: 균형 패널 데이터를
사용해 분석이 간결하고 해석이 명확합니다. 모든 국가가 동일한 연도
데이터를 가지므로 결과의 비교가 더 직관적입니다. 단점: 일부 국가의
데이터를 제외하면 샘플 크기가 줄어들어 통계적 검정력이 약해질 수
있습니다.

\begin{enumerate}
\def\labelenumi{\arabic{enumi}.}
\tightlist
\item
  Overall Model Insights Dependent Variable: Poverty Headcount Ratio
  (poverty rate).
\end{enumerate}

Independent Variables:

Financial inclusion indicators: account ownership, savings, borrowing,
digital payments. Economic growth variable: Real GDP Growth.
Observations: A total of 141 data points.

R² (Explained Variance):

The model explains 17.2\% of the variance in poverty rates. Adjusted R²
is negative (-0.302), indicating the model struggles to explain the data
when accounting for the number of predictors. Conclusion: The model has
low explanatory power, suggesting limitations in variable selection or
data structure. F-Statistic: 3.703*** (p \textless{} 0.01):

The overall model is statistically significant, indicating that at least
one independent variable is related to the dependent variable. 2.
Interpretation of Individual Variables (a) Account Ownership (Income,
poorest 40\%) Coefficient: 0.505 Standard Error: (3.344) p-value: Not
significant. Interpretation: Account ownership for the poorest 40\%
shows a positive relationship with poverty rates (poverty increases as
account ownership rises), but the result is not statistically
significant. No robust conclusion can be drawn about the effect of
account ownership on poverty. (b) Savings (Income, poorest 40\%)
Coefficient: 0.757 Standard Error: (1.200) p-value: Not significant.
Interpretation: Savings behavior shows a small positive relationship
with poverty rates, but this result is not statistically significant.
Savings activity does not have a clear impact on reducing poverty in
this dataset. (c) Borrowing (Income, poorest 40\%) Coefficient:
3.524\emph{ Standard Error: (1.807) p-value: }p \textless{} 0.1
(significant). Interpretation: Borrowing is positively and significantly
associated with poverty rates, suggesting that an increase in borrowing
correlates with an increase in poverty. This could indicate that
borrowing is not being used effectively or that it leads to higher debt
burdens among the poorest populations. (d) Digital Payments (Income,
poorest 40\%) Coefficient: -3.146 Standard Error: (2.595) p-value: Not
significant. Interpretation: Digital payment use shows a weak negative
relationship with poverty rates (poverty decreases as digital payments
rise), but the result is not statistically significant. The impact of
digital payments on poverty reduction is unclear. (e) Real GDP Growth
Coefficient: -0.004 Standard Error: (0.034) p-value: Not significant.
Interpretation: Economic growth shows virtually no relationship with
poverty rates and is statistically insignificant. This suggests that
economic growth alone is insufficient to address poverty in the studied
context. 3. Key Conclusions Significant Variable:

Borrowing behavior (Borrowing (Income, poorest 40\%)) is significantly
associated with poverty increases, indicating potential issues with debt
management among the poorest groups. Non-significant Variables:

Other financial inclusion variables (account ownership, savings, digital
payments) and GDP growth do not show significant relationships with
poverty. This might reflect data limitations or that these factors have
limited direct effects on poverty rates in the studied context. Model's
Explanatory Power:

The low R² value suggests that the model does not sufficiently capture
the drivers of poverty. Including additional variables such as
education, infrastructure, or social policies could improve explanatory
power. Policy Implications:

Effective borrowing management and financial education programs could
help alleviate poverty among the poorest groups. Other financial
inclusion activities might need complementary policies to be impactful.

\end{document}
